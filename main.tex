\documentclass{article}
\usepackage[utf8]{inputenc}
\usepackage[brazilian]{babel}
\usepackage{paracol}
\usepackage{hyperref}
\usepackage{vmargin} 
\setpapersize{A4}
\setmargins{2.5cm} 
{1.5cm} 
{16.5cm} 
{23.42cm} 
{10pt} 
{1cm} 
{0pt} 
{1cm} 

\usepackage{xcolor}
\usepackage{tcolorbox}
\tcbuselibrary{listings, skins}
\usepackage{listings}

\definecolor{arduino}{HTML}{00A3A9}
\definecolor{structure}{HTML}{818A42}
\definecolor{variables}{HTML}{128F8F}
\definecolor{functions}{HTML}{DB6B21}
\definecolor{back}{HTML}{E0E0E2}
\definecolor{myblue}{rgb}{0.01,0.61,0.98}
\definecolor{mygray}{rgb}{0.47,0.47,0.33}

\newcommand*{\FormatDigit}[1]{\ttfamily\textcolor{black}{#1}}
%% http://tex.stackexchange.com/questions/32174/listings-package-how-can-i-format-all-numbers
%% Correção de erros em acentos e caracteres especiais da língua portuguesa:
\lstdefinestyle{FormattedNumber}{%
    literate={á}{{\'a}}{1}%
            {à}{{\`a}}{1}%
            {ã}{{\~a}}{1}%
            {â}{{\^a}}{1}%
            {é}{{\'e}}{1}%
            {è}{{\`e}}{1}%
            {ê}{{\^e}}{1}%
            {í}{{\'i}}{1}%
            {ó}{{\'o}}{1}%
            {ô}{{\^o}}{1}%
            {ò}{{\`o}}{1}%
            {õ}{{\~o}}{1}%
            {ú}{{\'u}}{1}%
            {ç}{{\c{c}}}{1}%
            {Á}{{\'A}}{1}%
            {À}{{\`A}}{1}%
            {Â}{{\^A}}{1}%
            {Ã}{{\~A}}{1}%
            {É}{{\'E}}{1}%
            {È}{{\`E}}{1}%
            {Ê}{{\^E}}{1}%
            {Í}{{\'I}}{1}%
            {Ó}{{\'O}}{1}%
            {Ô}{{\^O}}{1}%
            {Ò}{{\`O}}{1}%
            {Õ}{{\~O}}{1}%
            {Ú}{{\'U}}{1}%
            {Ç}{{\c C}}{1}%
            {0}{{\FormatDigit{0}}}{1}%
            {1}{{\FormatDigit{1}}}{1}%
            {2}{{\FormatDigit{2}}}{1}%
            {3}{{\FormatDigit{3}}}{1}%
            {4}{{\FormatDigit{4}}}{1}%
            {5}{{\FormatDigit{5}}}{1}%
            {6}{{\FormatDigit{6}}}{1}%
            {7}{{\FormatDigit{7}}}{1}%
            {8}{{\FormatDigit{8}}}{1}%
            {9}{{\FormatDigit{9}}}{1}%
            {.0}{{\FormatDigit{.0}}}{2}% 
            {.1}{{\FormatDigit{.1}}}{2}% 
            {.2}{{\FormatDigit{.2}}}{2}%
            {.3}{{\FormatDigit{.3}}}{2}%
            {.4}{{\FormatDigit{.4}}}{2}%
            {.5}{{\FormatDigit{.5}}}{2}%
            {.6}{{\FormatDigit{.6}}}{2}%
            {.7}{{\FormatDigit{.7}}}{2}%
            {.8}{{\FormatDigit{.8}}}{2}%
            {.9}{{\FormatDigit{.9}}}{2}%
            %{,}{{\FormatDigit{,}}}{1}% Eliminar o comentário se quiser "," colorido
            {\ }{{ }}{1}%
            ,%
}

\lstset{%
  basicstyle=\footnotesize,       
  breakatwhitespace=false,         
  breaklines=true,                 
  captionpos=b,                   
  commentstyle=\color{gray},    
  deletekeywords={...},           
  escapeinside={\%*}{*)},          
  extendedchars=true,              
  keepspaces=true,                 
  keywordstyle=[1]\color{structure},
  keywordstyle=[2]\color{variables},
  keywordstyle=[3]\color{functions},
  keywordstyle=[4]\bfseries\color{functions},
  language=c++,                
  morekeywords={*,...},     
  numbers=left,                    
  numbersep=5pt,                   
  numberstyle=\tiny\color{mygray}, 
  rulecolor=\color{black},         
  rulesepcolor=\color{myblue},
  showspaces=false,                
  showstringspaces=false,          
  showtabs=false,                
  stringstyle=\color{rgb: red,0.33;green,0.45;blue,0.87},    
  tabsize=2,                       
  title=\lstname,
  emphstyle=\color{variables},
  frame = single,
  framexleftmargin = 15pt,
  rulecolor = \color{arduino},
}

\lstdefinestyle{Arduino}{%
    style=FormattedNumber,
    keywords={setup, loop, if, else, for, switch, while, do, break, continue, return, goto},
    morekeywords=[2]{HIGH, LOW, INPUT, OUTPUT, INPUT_PULLUP, LED_BUILTIN, true, false, int, float, void, boolean, char, word, long, short, double, string, array},
    morekeywords=[3]{const, pinMode, digitalWrite, digitalRead, analogReference, analogRead, analogWrite, analogReadResolution, analogWriteResolution, tone, noTone, shiftOut, shiftIn, pulseIn, millis, micros, delay, delayMicroseconds, min, max, abs, constrain, map, pow, sqrt, sin, cos, tan, isAlphaNumeric, inAlpha, isAscii, isWhitespace, isControl, isDigit, isGraph, isLowerCase, isPintable, isPunct, isSpace, isUpperCase, isHexadecimalDigit, randomSeed, random, lowByte, highByte, bitRead, bitWrite, bitSet, bitClear, bit, attachInterrupt, detachInterrupt, interrupts, noInterrupts, Stream, Keyboard, Mouse, begin, println, print},
    morekeywords=[4]{Serial},
    morecomment=[l]{//},
    morecomment=[s]{/*}{*/},
    emph={const},
}

% Comando para incluir um sketch de Arduino, o primeiro parâmetro é o nome do arquivo que contem o script (sin .ino), o segundo é a legenda do contador Listing
\newcommand{\ArduinoSketch}[2]{
\begin{itemize}
\item[]\lstinputlisting[caption=#2,label=#1,style=Arduino]{#1.ino}
\end{itemize}
}

% Ambiente para incluir um sketch de Arduino escrevendo o código diretamente no documento LaTeX, tem um parâmetro de entrada que corresponde ao título do sketch
\newtcblisting{ArduinoSketchBox}[2][colframe = arduino, enhanced, drop shadow, hbox]{
	arc = 3pt, outer arc = 3pt,
	listing only,
	listing options = {
		frame =,
		style = Arduino,
	},
	title = #2,
	#1
}
%-----------------------------------------------------------------------------

\input{latex_code.tex}

\title{Incluindo códigos Arduino em documentos \LaTeX{}}
\author{Gabriel Borralho}
\date{\today}

\begin{document}
\maketitle

O arquivo \bftt{arduino\_code.tex} contém a definição do comando \cmdbs{ArduinoSketch} e o ambiente \linebreak \cmd{ArduinoSketchBox} que usam o pacote \bftt{listings} para definir o realce de sintaxe do Arduino IDE, o comando \bftt{ArduinoSketch} carrega o código de um esboço do Arduino do arquivo \bftt{.ino}, por isso é adequado para comprimentos de esboço que podem ocupar mais de uma página do documento, por outro lado, o ambiente \bftt{ArduinoSketchBox} usa o código do Arduino dentro de uma caixa colorida que pode ser ajustada dentro de uma coluna ou um pequeno espaço dentro do texto, é adequado para pequenos esboços que não ocupam mais de uma página ou para destacar partes do código.

O realce de sintaxe usado no arquivo \bftt{arduino\_code.tex} pode ser visto em \url{https://www.arduino.cc/en/Reference/HomePage}.

\section{Modo de exemplo de arquivo \bftt{arduino\_code.tex}}

Para começar a incluir o código do Arduino em um documento \LaTeX{} simplesmente copie o arquivo \bftt{arduino\_code.tex} para o diretório raiz do seu documento e use o comando \cmdbs{input} no preâmbulo


\begin{minted}[frame = single]{latex}
\input{Arduino_code.tex}
\end{minted}

\section{Modo de uso do ambiente \cmd{ArduinoSketchBox}}

\begin{minted}[frame = single]{latex}
\begin{ArduinoSketchBox}{title}
Código...
\end{ArduinoSketchBox}
\end{minted}

\begin{paracol}{2}
O ambiente \bftt{ArduinoSketchBox} não carrega o código do esboço do arquivo \bftt{.ino}, portanto é aconselhável usar \textit{skets} curtos ou para destacar partes do esboço geral, use um parâmetro de entrada correspondente ao título do esboço. esboço, neste exemplo, foi usado em combinação com o pacote \bftt{paracol} para incluir código Arduino em uma das colunas.
\switchcolumn
\begin{ArduinoSketchBox}{Ler o valor de um potenciômetro}
int sensor = A0;
int valSensor = 0;

void setup(){
	Serial.begin(9600);
}

void loop(){
	valSensor = analogRead(sensor);
    Serial.println(valSensor);
    delay(100);
}
\end{ArduinoSketchBox}
\end{paracol}

\newpage

\section{Exemplo do comando \cmdbs{ArduinoSketch} de um código deum leitor RFID}

A sintaxe de uso do comando \cmdbs {ArduinoSketch} é muito simples, pois o comando usa dois parâmetros de entrada \ bftt {file.ino} que não devem ter a extensão \bftt{.ino} e \bftt{Label para o código} correspondente à \bftt{caption} do contador \bftt{Listing}.

\begin{minted}[frame = single]{latex}
\ArduinoSketch{archivo.ino}{Leitor RFID}
\end{minted}

Como visto neste exemplo de uso, se o código usa mais espaço que a página atual, o quadro que o contém é dividido automaticamente.

\ArduinoSketch{RFID}{RFID}

A seguir temos a ilustração do circuito para o código implementado:

\begin{figure}[htb]
	\centering
	\includegraphics[scale=0.2]{Circuito.png}
	\caption{Ilustração do Circuito leitor RFID}
\end{figure}
\end{document}